% !TEX encoding = UTF-8 Unicode
\documentclass[a4paper]{article}

\usepackage{color}
\usepackage{url}
\usepackage[T2A]{fontenc} % enable Cyrillic fonts
\usepackage[utf8]{inputenc} % make weird characters work
\usepackage{graphicx}

\usepackage[english,serbian]{babel}
%\usepackage[english,serbianc]{babel} %ukljuciti babel sa ovim opcijama, umesto gornjim, ukoliko se koristi cirilica

\usepackage[unicode]{hyperref}
\hypersetup{colorlinks,citecolor=green,filecolor=green,linkcolor=blue,urlcolor=blue}

\usepackage{listings}

%\newtheorem{primer}{Пример}[section] %ćirilični primer
\newtheorem{primer}{Primer}[section]

\definecolor{mygreen}{rgb}{0,0.6,0}
\definecolor{mygray}{rgb}{0.5,0.5,0.5}
\definecolor{mymauve}{rgb}{0.58,0,0.82}

\lstset{ 
  backgroundcolor=\color{white},   % choose the background color; you must add \usepackage{color} or \usepackage{xcolor}; should come as last argument
  basicstyle=\scriptsize\ttfamily,        % the size of the fonts that are used for the code
  breakatwhitespace=false,         % sets if automatic breaks should only happen at whitespace
  breaklines=true,                 % sets automatic line breaking
  captionpos=b,                    % sets the caption-position to bottom
  commentstyle=\color{mygreen},    % comment style
  deletekeywords={...},            % if you want to delete keywords from the given language
  escapeinside={\%*}{*)},          % if you want to add LaTeX within your code
  extendedchars=true,              % lets you use non-ASCII characters; for 8-bits encodings only, does not work with UTF-8
  firstnumber=1000,                % start line enumeration with line 1000
  frame=single,	                   % adds a frame around the code
  keepspaces=true,                 % keeps spaces in text, useful for keeping indentation of code (possibly needs columns=flexible)
  keywordstyle=\color{blue},       % keyword style
  language=Perl,                 % the language of the code
  morekeywords={*,...},            % if you want to add more keywords to the set
  numbers=left,                    % where to put the line-numbers; possible values are (none, left, right)
  numbersep=5pt,                   % how far the line-numbers are from the code
  numberstyle=\tiny\color{mygray}, % the style that is used for the line-numbers
  rulecolor=\color{black},         % if not set, the frame-color may be changed on line-breaks within not-black text (e.g. comments (green here))
  showspaces=false,                % show spaces everywhere adding particular underscores; it overrides 'showstringspaces'
  showstringspaces=false,          % underline spaces within strings only
  showtabs=false,                  % show tabs within strings adding particular underscores
  stepnumber=2,                    % the step between two line-numbers. If it's 1, each line will be numbered
  stringstyle=\color{mymauve},     % string literal style
  tabsize=2,	                   % sets default tabsize to 2 spaces
  title=\lstname                   % show the filename of files included with \lstinputlisting; also try caption instead of title
}

\begin{document}
\section{Instairanje i pokretanje}
Da bi se uspe\v sno pokretao perl skript potrebno je imati kompajler. U ovom poglavlju bi\'ce obja\v snjeno njegovo instaliranje na Linux i Windows operativnim sistemima.

\subsection{Instaliranje na Linux operativnom sistemu}
Da bismo instalirali Perl kompajler na Linux operativnom sistemu potrebno je da u terminalu unesemo komandu sudo apt-get install perl. Od vas \'ce se zatra\v ziti da unesete lozinku. Kada se instalira Perl kompajler u terminalu pokrenite slede\'cu komandu curl -L http://xrl.us/installperlnix | bash. Kada sve bude gotovo, da biste se uverili da je instalacija uspe\v sno izvr\v sena ili ako \v zelite da proverite koja je verzija Perl-a instalirana, to mo\v zete uraditi komandom perl -v.

\subsection{Pokretanje na Linux operativnom sistemu}
Pokretanje Perl skripta na Linux operativnom sistemu se radi na jednostavan na\v cin. Potrebno je otvoriti terminal, a zatim uneti komandu perl <putanja\_do\_fajla>. Ova komanda \'ce pokrenuti izvr\v savanje skripta.

\subsection{Instairanje na Windows operativnom sistemu}
Pre instalacije na Windows operativnom sistemu po\v zeljno je  prethodno proveriti da nijedna verzija Perla ve\'c nije instalirana. Ako \v zelite da deinstalirate prethodnu verziju idite na Control Panel - Add/Remove programs. Ako i dalje imate C:\textbackslash Strawberry folder, izbri\v site ga ili ga preimenujte. Skinite i instalirajte Padre, the Perl IDE/editor\cite{padre}. Strawberry Perl\cite{strawberry} je deo instalacije, ali tako\dj{}e dobijate i mnoge druge korisne CPAN module. Nakon instalacije, mo\v zda \'cete morati da restartujete ra\v cunar. Zatim idite na Start meni/All Programs i prona\dj{}ite folder sa Perlom. U njemu kliknite na Perl komandnu liniju.  Da biste instalirali module iz CPAN-a u komandnoj liniji ukucajte komandu cpan App::cpanminus. Ukoliko \v zeite da  potvrdite da je instalacija uspe\v sna ukucajte perl -v, ovom komandom, tako\dj{}e, mo\v zete proveriti koja je verzija perl-a instalirana.

\subsection{Pokretanje na Windowsu}
Ako imamo instaliran Padre, the Perl IDE/editor, komande za pokretanje programa mo\v zemo zadati i iz editora. U meniju izaberemo Run, a zatim Run Script, mada mo\v zemo da pritisnemo taster F5 i skript \'ce biti pokrenut.
Ako \v zelimo, skript mo\v zemo pokrenuti i iz komandne linije tako \v sto ukucamo perl <putanja\_do\_fajla>.

\section{Primeri koda}
U listingu \ref{hello} dat je primer ispisa Hello world programa. Komentare pi\v semo iza tarabe. Tekst mo\v zemo ispisati tako \v sto ga navedemo iza navodnika ili koriste\'ci funkciju qq.
\begin{lstlisting}[caption={Primer Hello world programa},frame=single, label=hello]
#Komentar
print "Hello, world!\n";
print qq =Did you say "Hello"?\n=;
\end{lstlisting}

Naredni primer demonstrira program koji ra\v cuna zbir dva broja. Imena promenljivih po\v cinju znakom \$. U komentarima je dato \v sta koja print naredba ispisuje.
\begin{lstlisting}[caption={Zbir dva broja}, frame=single, label = zbir]
$a = 5;
$b = 4;
print qq =a \= $a\n=; # a = 5
print qq =b \= $b\n=; # b = 4
$zbir = $a + $b;
print qq=a + b \= $zbir\n=; #a + b = 9
\end{lstlisting}

Nazivi nizova u Perlu su u formatu @naziv\_niza. Niz se mo\v ze inicijalizovati tako \v sto elemete navedemo u zagradama, razdvojene zarezom. Ako \v zelimo da ispi\v semo sve elemete niza, to mo\v zemo u\v ciniti samo navo\dj{}enjem imena niza. Odre\dj{}enom \v clanu niza pristupamo tako \v sto njegov indeks navedemo u uglastim zagradama iza naziva niza. U ovom su\v caju ime niza mo\v ze po\v cinjati i znakom @ i znakom \$. Indeks poslednjeg \v clana niza dobijamo tako \v sto ispre imena niza stavimo \$\#.
\begin{lstlisting}[caption={Niz}, frame=single, label=niz]
@dan = ("Danas", "je", "lep", "dan"); 
$broj_reci = @dan; #Broje elementa niza
print "broj reci u recenici \"@dan\" je $broj_reci.\n";
print "@dan[2], $dan[3]\n";
print "Indeks posednjeg elementa je $#dan\n";
\end{lstlisting}
Grananje u programu mo\v zemo izvrsiti koriste\'ci if-else naredbu. Naredni program prvo u\v citava broj sa standardnog ulaza, za zatim proverava da li je pozitivan ili je negativan. Uslov se navodi u zagradi, a if i else blok se navode izma\dj{} viti\v castih zagrada.
\begin{lstlisting}[caption={If-else naredba}, frame=single, label=ifelse]
$broj = <STDIN>;

if($broj >= 0){
    print "Broj je pozitivan\n";
}
else{
    print "Broj je negativan\n";
}
\end{lstlisting}

Mo\v zemo zadavati i argumente komandne linije. Argumentima komandne linije pristupamo isto kao i elementima niza, samo \v sto kao naziv niza navodimo ARGV. Broj argumenata komandne linije mo\v zemo dobiti tako \v sto indeks poslednjeg elementa uve\'camo za jedan.
\begin{lstlisting}[caption={Argumenti komandne linije}, frame=single, label=argv]
$br_arg = $#ARGV + 1;
print "Broj argumenata komandne linije je $br_arg, a to su @ARGV\n";
\end{lstlisting}
Slede\'cim kodom je prikazan primer kori\v s\'cenja while petlje. Program prihvata argumente sa ulaza i ispisuje ih na ekran. Klju\v cna re\v c my\cite{my} ozna\v cava da je domen promenljive samo taj blok. Promenljiva \$in u while petlji nema uticaj na promenljivu sa istim nazivom van petlje.
\begin{lstlisting}[caption={Unos i ispis elemenata u while petlji}, frame=single, label=while]
$in = 5;

while(my $in = <>) {
    print $in; #Ispisuje element koji smo uneli
}
print $in; #Ispisuje broj 5
\end{lstlisting}
Prethodnu petlju mo\v zemo zapisati i na slede\'ci na\v cin.
\begin{lstlisting}[caption={Unos i ispis elemenata u while petlji}, frame=single, label=while1]
while(<>) {
    print;
}
\end{lstlisting}
U Perlu mo\v zemo koristiti i for petlju na sli\v can na\v cin kao i u programskom jeziku C.
\begin{lstlisting}[caption={For petlja}, frame=single, label=for]
for(my $i = 0; $i <= $#ARGV; $i++) {
    print "$ARGV[$i]\n";
}
\end{lstlisting}
Funkcije zapo\v cinju klju\v cnom re\v cju sub iza koje sledi naziv funkcije. Telo funkcije se pi\v se izme\dj{}u viti\v castih zagrada. Argumenti se navode u telu funkcije na slede\'ci na\v cin my(lista\_argumenata) = @\_; Poziv funkcije se vr\v si na standardan na\v cin.
\begin{lstlisting}[caption={Funkcija koja ra\v cuna obim kvadrata}, frame=single, label=funkcija]
sub obim_kvadrata {
    my($a) = @_;
    
    return 4*$a;
}

$obim = obim_kvadrata($aa);
print "$obim\n"; 

print obim_kvadrata(4);
print "\n";
\end{lstlisting}

\begin{primer}
Slede\'ci kod predstavlja jednostavan na\v cin da zamenimo vrednosti dve promenljive.
\begin{lstlisting}[caption={Zamena vrednosti dve promenljive}, frame=single, label=zamena]
aa = 3;
$bb = 2;
($aa, $bb) = ($bb, $aa);
print "$aa $bb\n";

\end{lstlisting}

\end{primer}

\bibliography{instalacija.bib}
\bibliographystyle{abbrv}


\end{document}
